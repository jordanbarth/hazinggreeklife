\documentclass[12pt]{amsart}
\usepackage{geometry} % see geometry.pdf on how to lay out the page. There's lots.
\geometry{a4paper} % or letter or a5paper or ... etc
% \geometry{landscape} % rotated page geometry

% See the ``Article customise'' template for come common customisations

\title{Hazing in Greek Life}
\author{Jordan Barth}
\date


 % delete this line to display the current date

%%% BEGIN DOCUMENT
\begin{document}

\maketitle

\section{Introduction}

Hazing on college campuses and more specifically, greek organizations, has received much press attention as of late. Just last year, Tim Piazza (Penn State), Max Gruver (Louisiana State University) and (Texas State) died from actions (or lack their of) taken by fraternity "brothers." As a result, Fraternities have been suspended, former members now face criminal charges, and greek life at these universities have undergone substantial change. 

\section{Defining the Problem}

While not limited to greek organizations, hazing been covered most when occurring in greek life. Why hazing occurs is complex and often differs by organization. Its often a combination of social psychology, adolescent development, and cultural norms that provides ripe conditions for greek organizations to haze new members. Long standing traditions exist in greek organizations that with little oversight, go way too far. In addition, adolecents are unable to fully envision the consequesnces of their actions (without alcohol). combined with alcohol-induced adolescent maturing throughout college combined with groupthink

those who are hazed and those doing the hazing may begin by participating in a relatively innocuous activity and then find themselves moving along a slippery slope toward more extreme behavior. Escalation of commitment and a desire to act in ways that are consistent with prior decisions and behavior may all contribute to making it difficult to find another path. The illusion of control and overconfidence may mean that participants discount the possibility of hazing's having bad outcomes.

Social conformity, obedience to authority, organizational identification, group effects and the need for acceptance, connectedness and esteem may help to explain the persistence of hazing rituals. Even participants who are disturbed by hazing rituals may not speak out, due in part to pluralistic ignorance, or the misperception of social norms. Having survived hazing rituals, participants may feel a sense of accomplishment, experience the respect of their peers, and experience cognitive dissonance that distorts their memories of their experiences.

\subsection{Social Psychology and Cultural Norms}



\subsection{Adolescent Development}





\section{Data and Methodology}

Measuring hazing is difficult. It is one of the few areas with no clearinghouse or hub to pull data from.  o clearinghouse/data hub to pull information from -- all done by google search

Data Sources: 1) Office of Campus Life, 2) Campus, Local, Regional and National Publications
Time of Interest: 2015-2018

Methodological challenges- While this only looks at hazing, heavy alcohol usage usually is part of hazing rituals, some universities consider it a separate violation others don?t. (same with health and safety) 
No clearinghouse/data hub to pull information from -- all done by google search
Some universities post status of greek organizations but its often incomplete or quite scattered

In order to classify greek organizations have recieved, I've semi-operationalized the sanctions by including a breakdown below. 

\begin{itemize}
  \item Cease and Desist: If an investigation is being conducted or a hearing with the Committee on Student Life (CSL) is pending, a chapter may be put on a temporary Cease and Desist status. Cease and desist orders require the chapter to temporarily stop all activity, including but not limited to individual initiated member/new member gatherings, recruitment events, social events, and new member meetings. Any necessary fraternity business meetings may be pre-approved by the OFSL.
  \item Disciplinary Probation: An organization is not in good standing with the University after having been found responsible for serious violations of University policy or subsequent violations after prior disciplinary action. This may be determined by a hearing through the IGC Judiciary or the CSL. Violations of the terms of Disciplinary Probation, which are specified in the chapter?s outcome letter, or any new violations of University policy while the chapter is on probation may result in the revocation of the organization?s recognition.  Organizations on disciplinary probation are permitted to conduct operations as usual unless additional modifications to operating status are made in the chapter outcome letter.
  \item Social Probation: The organization is unable to host social events with alcohol. During social probation, chapters are not permitted to coordinate and/or hold social events with alcohol. This includes social functions with alcohol in the chapter house, brothers' homes, or third party venues. Social probation has no limit to minimum or maximum duration. Recruitment, new member education, educational events, and philanthropy events are allowed during social probation.
  \item Suspension: The organization's recognition is suspended by the University for a specified period of time. The suspension may be deferred provided that the organization complete the requirements outlined from a disciplinary hearing. Further violations could result in the permanent revocation of the organization?s University recognition.
\end{itemize}


\section{Results and Analysis}

\end{document}